\documentclass[a4paper,10pt]{article}
\usepackage[utf8]{inputenc}
\usepackage{amsmath}

%opening
\title{Artificial Neural Networks:\\ Exercise Set 4}
\author{Antonio Peters}

\begin{document}

\maketitle

\section{Question 3}
As shown by the data given, new inputs which vary from the original $P$ generate the same output as that of $P$, even though this is incorrect for the new data, it is due to the fact that the network has been trained to generate the $T$ of $P$ and not on the relationship between $P$ and $T$.

\section{Question 9}
If there is only one pattern and one target then our equation
\begin{equation}
 t = f(Wq+b)
\end{equation}
need only fit the target perfectly, with no need to interpolate new data. Since $t$,$f$ and $q$ are already known, all that is needed is to solve for $W$ and $b$. $W$ and $b$ can be combined into $V$ where $V=[W b]$ by appending $p$ into $q$ with $q=[p;1]$. This then transforms the equation into 
\begin{equation}
 t = f(Vq)
\end{equation}
Thereby leaving us with only one unknown, $V$. The equation can then be easily transformed to solve for $V$
\begin{equation}
\begin{align}
 t &= f(Vq)\\
 Vq &= f^{-1}(t)\\
 V &= f^{-1}(t)/q\\
\end{align}
\end{equation}
Since this equation has an exact solution, a perfect fitting from $p$ to $t$ can be found by solving for $V$. As stated before, this solution will be ill-advised for a neural network as it only fits these data points and cannot properly equate any new points in the system.

\section{Question 10}
$2a$ has the best time, averaging 4 iterations while $2b$ and $3b$ have 7, $3a$ takes 10. $1a$ and $1b$ fail to converge as the number of neurons in the layer are too few.

\section{Question 11}
With the given data the solution converges but never perfectly to the target within 100 iterations.
\end{document}
